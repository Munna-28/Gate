\let\negmedspace\undefined
\let\negthickspace\undefined
\def\inputGnumericTable{} 
\documentclass[journal,12pt,twocolumn]{IEEEtran}
%\documentclass[conference]{IEEEtran}
%\IEEEoverridecommandlockouts
% The preceding line is only needed to identify funding in the first footnote. If that is unneeded, please comment it out.
\usepackage{cite}
\usepackage{amsmath,amssymb,amsfonts,amsthm}
\usepackage{algorithmic}
\usepackage{graphicx}
\usepackage{textcomp}
\usepackage{xcolor}
\usepackage{txfonts}
\usepackage{listings}
\usepackage{enumitem}
\usepackage{mathtools}
\usepackage{gensymb}
\usepackage[breaklinks=true]{hyperref}
\usepackage{tkz-euclide} % loads  TikZ and tkz-base
\usepackage{listings}
\usepackage{caption}
\usepackage[latin1]{inputenc}                                 
    \usepackage{color}                                            
    \usepackage{array}                                            
    \usepackage{longtable}                                        
    \usepackage{calc}                                             
    \usepackage{multirow}                                         
    \usepackage{hhline}                                           
    \usepackage{ifthen}

%
%\usepackage{setspace}
%\usepackage{gensymb}
%\doublespacing
%\singlespacing

%\usepackage{graphicx}
%\usepackage{amssymb}
%\usepackage{relsize}
%\usepackage[cmex10]{amsmath}
%\usepackage{amsthm}
%\interdisplaylinepenalty=2500
%\savesymbol{iint}
%\usepackage{txfonts}
%\restoresymbol{TXF}{iint}
%\usepackage{wasysym}
%\usepackage{amsthm}
%\usepackage{iithtlc}
%\usepackage{mathrsfs}
%\usepackage{txfonts}
%\usepackage{stfloats}
%\usepackage{bm}
%\usepackage{cite}
%\usepackage{cases}
%\usepackage{subfig}
%\usepackage{xtab}
%\usepackage{longtable}
%\usepackage{multirow}
%\usepackage{algorithm}
%\usepackage{algpseudocode}
%\usepackage{enumitem}
%\usepackage{mathtools}
%\usepackage{tikz}
%\usepackage{circuitikz}
%\usepackage{verbatim}
%\usepackage{tfrupee}
%\usepackage{stmaryrd}
%\usetkzobj{all}
%    \usepackage{color}                                            %%
%    \usepackage{array}                                            %%
%    \usepackage{longtable}                                        %%
%    \usepackage{calc}                                             %%
%    \usepackage{multirow}                                         %%
%    \usepackage{hhline}                                           %%
%    \usepackage{ifthen}                                           %%
  %optionally (for landscape tables embedded in another document): %%
%    \usepackage{lscape}     
%\usepackage{multicol}
%\usepackage{chngcntr}
%\usepackage{enumerate}

%\usepackage{wasysym}
%\newcounter{MYtempeqncnt}
\DeclareMathOperator*{\Res}{Res}
%\renewcommand{\baselinestretch}{2}
\renewcommand\thesection{\arabic{section}}
\renewcommand\thesubsection{\thesection.\arabic{subsection}}
\renewcommand\thesubsubsection{\thesubsection.\arabic{subsubsection}}

\renewcommand\thesectiondis{\arabic{section}}
\renewcommand\thesubsectiondis{\thesectiondis.\arabic{subsection}}
\renewcommand\thesubsubsectiondis{\thesubsectiondis.\arabic{subsubsection}}

% correct bad hyphenation here
\hyphenation{op-tical net-works semi-conduc-tor}
\def\inputGnumericTable{}                                 %%

\lstset{
%language=C,
frame=single, 
breaklines=true,
columns=fullflexible
}
%\lstset{
%language=tex,
%frame=single, 
%breaklines=true
%}

\begin{document}
%


\newtheorem{theorem}{Theorem}[section]
\newtheorem{problem}{Problem}
\newtheorem{proposition}{Proposition}[section]
\newtheorem{lemma}{Lemma}[section]
\newtheorem{corollary}[theorem]{Corollary}
\newtheorem{example}{Example}[section]
\newtheorem{definition}[problem]{Definition}
%\newtheorem{thm}{Theorem}[section] 
%\newtheorem{defn}[thm]{Definition}
%\newtheorem{algorithm}{Algorithm}[section]
%\newtheorem{cor}{Corollary}
\newcommand{\BEQA}{\begin{eqnarray}}
\newcommand{\EEQA}{\end{eqnarray}}
\newcommand{\define}{\stackrel{\triangle}{=}}

\bibliographystyle{IEEEtran}
%\bibliographystyle{ieeetr}


\providecommand{\mbf}{\mathbf}
\providecommand{\pr}[1]{\ensuremath{\Pr\left(#1\right)}}
\providecommand{\qfunc}[1]{\ensuremath{Q\left(#1\right)}}
\providecommand{\sbrak}[1]{\ensuremath{{}\left[#1\right]}}
\providecommand{\lsbrak}[1]{\ensuremath{{}\left[#1\right.}}
\providecommand{\rsbrak}[1]{\ensuremath{{}\left.#1\right]}}
\providecommand{\brak}[1]{\ensuremath{\left(#1\right)}}
\providecommand{\lbrak}[1]{\ensuremath{\left(#1\right.}}
\providecommand{\rbrak}[1]{\ensuremath{\left.#1\right)}}
\providecommand{\cbrak}[1]{\ensuremath{\left\{#1\right\}}}
\providecommand{\lcbrak}[1]{\ensuremath{\left\{#1\right.}}
\providecommand{\rcbrak}[1]{\ensuremath{\left.#1\right\}}}
\theoremstyle{remark}
\newtheorem{rem}{Remark}
\newcommand{\sgn}{\mathop{\mathrm{sgn}}}
\providecommand{\abs}[1]{\left\vert#1\right\vert}
\providecommand{\res}[1]{\Res\displaylimits_{#1}} 
\providecommand{\norm}[1]{\left\lVert#1\right\rVert}
%\providecommand{\norm}[1]{\lVert#1\rVert}
\providecommand{\mtx}[1]{\mathbf{#1}}
\providecommand{\mean}[1]{E\left[ #1 \right]}
\providecommand{\fourier}{\overset{\mathcal{F}}{ \rightleftharpoons}}
%\providecommand{\hilbert}{\overset{\mathcal{H}}{ \rightleftharpoons}}
\providecommand{\system}{\overset{\mathcal{H}}{ \longleftrightarrow}}
	%\newcommand{\solution}[2]{\textbf{Solution:}{#1}}
\newcommand{\solution}{\noindent \textbf{Solution: }}
\newcommand{\cosec}{\,\text{cosec}\,}
\providecommand{\dec}[2]{\ensuremath{\overset{#1}{\underset{#2}{\gtrless}}}}
\newcommand{\myvec}[1]{\ensuremath{\begin{pmatrix}#1\end{pmatrix}}}
\newcommand{\mydet}[1]{\ensuremath{\begin{vmatrix}#1\end{vmatrix}}}
%\numberwithin{equation}{section}
%\numberwithin{equation}{subsection}
%\numberwithin{problem}{section}
%\numberwithin{definition}{section}
%\makeatletter
%\@addtoreset{figure}{problem}
%\makeatother

%\let\StandardTheFigure\thefigure
\let\vec\mathbf
%\renewcommand{\thefigure}{\theproblem.\arabic{figure}}
%\renewcommand{\thefigure}{\theproblem}
%\setlist[enumerate,1]{before=\renewcommand\theequation{\theenumi.\arabic{equation}}
%\counterwithin{equation}{enumi}


%\renewcommand{\theequation}{\arabic{subsection}.\arabic{equation}}

%\def\putbox#1#2#3{\makebox[0in][l]{\makebox[#1][l]{}\raisebox{\baselineskip}[0in][0in]{\raisebox{#2}[0in][0in]{#3}}}}
%     \def\rightbox#1{\makebox[0in][r]{#1}}
%     \def\centbox#1{\makebox[0in]{#1}}
%     \def\topbox#1{\raisebox{-\baselineskip}[0in][0in]{#1}}
%     \def\midbox#1{\raisebox{-0.5\baselineskip}[0in][0in]{#1}}

\bibliographystyle{IEEEtran}


\vspace{3cm}

\title{
%	\logo{
ASSIGNEMNT-1 PROBABILITY
}
\author{ Katherapaka Nikhil$^{*}$% <-this % stops a space 
}	


\maketitle

\newpage

%\tableofcontents

\bigskip

\renewcommand{\thefigure}{\theenumi}
\renewcommand{\thetable}{\theenumi}

%\renewcommand{\theequation}{\theenumi}

%\begin{abstract}
%%\boldmath
%In this letter, an algorithm for evaluating the exact analytical bit error rate  (BER)  for the piecewise linear (PL) combiner for  multiple relays is presented. Previous results were available only for upto three relays. The algorithm is unique in the sense that  the actual mathematical expressions, that are prohibitively large, need not be explicitly obtained. The diversity gain due to multiple relays is shown through plots of the analytical BER, well supported by simulations. 
%
%\end{abstract}
% IEEEtran.cls defaults to using nonbold math in the Abstract.
% This preserves the distinction between vectors and scalars. However,
% if the journal you are submitting to favors bold math in the abstract,
% then you can use LaTeX's standard command \boldmath at the very start
% of the abstract to achieve this. Many IEEE journals frown on math
% in the abstract anyway.

% Note that keywords are not normally used for peerreview papers.
%\begin{IEEEkeywords}
%Cooperative diversity, decode and forward, piecewise linear
%\end{IEEEkeywords}



% For peer review papers, you can put extra information on the cover
% page as needed:
% \ifCLASSOPTIONpeerreview
% \begin{center} \bfseries EDICS Category: 3-BBND \end{center}
% \fi
%
% For peerreview papers, this IEEEtran command inserts a page break and
% creates the second title. It will be ignored for other modes.
%\IEEEpeerreviewmaketitle
Question 35.
Suppose that ($\vec{X_1},\vec{X_2},\vec{X_3}$) has $N_3(\vec{\mu},\vec{\Sigma})$ distrubution with 
$\mu = \begin{bmatrix}
0 \\
0\\
0
\end{bmatrix} \quad  
\Sigma=\begin{bmatrix}
2 & 2 & 1 \\
2 & 5 & 1 \\
1 & 1 & 1
\end{bmatrix} \\ $ Given that $\phi(-0.5)$=0.3085, where $\phi(.)$ denotes the cumulative distribution function of a standard normal random variable, $P \brak{ \brak{X_1-2X_2+2X_3}^2 < \frac{7}{2} }$ equals to \\
\solution \\
Let $\vec{Y} = \vec{X_1}-\vec{2X_2}+\vec{2X_3}$ and $\vec{a}$=$ \begin{bmatrix}
1 \\ -2 \\  2 
\end{bmatrix}$
\begin{align}
\mu_Y &= \vec{a}^T \vec{\mu}  \\
&= \begin{bmatrix} 1 & -2 & 2 \end{bmatrix} \begin{bmatrix} 0 \\ 0 \\ 0 \end{bmatrix}  \\
&= 0 \\
\sigma_Y^2 &= \vec{a}^T \vec{\Sigma} \vec{a} \\
&= \begin{bmatrix} 1 & -2 & 2 \end{bmatrix} \begin{bmatrix} 2 & 2 & 1 \\ 2 & 5 & 1 \\ 1 & 1 & 1 \end{bmatrix} \begin{bmatrix} 1 \\ -2 \\ 2 \end{bmatrix} \\
&= 1 \\ 
\pr{Y^2 < \frac{7}{2}} &= \pr{-\sqrt{\frac{7}{2}} < Y < \sqrt{\frac{7}{2}}}\\
% &= \Phi\myvec{\sqrt{\frac{7}{2}}}- \Phi\myvec{-\sqrt{\frac{7}{2}}} \\
\Phi(x) &= \int_{-\infty}^{x} \frac{1}{\sqrt{2\pi}}e^{-\frac{z^2}{2}} dz \\
\Phi\myvec{\sqrt{\frac{7}{2}}} &= \int_{-\infty}^{\sqrt{\frac{7}{2}}} \frac{1}{\sqrt{2\pi}}e^{-\frac{z^2}{2}} dz \\
&=\int_{-\infty}^{-\frac{1}{2}} \frac{1}{\sqrt{2\pi}}e^{-\frac{z^2}{2}} +\int_{-\frac{1}{2}}^{\sqrt{\frac{7}{2}}} \frac{1}{\sqrt{2\pi}}e^{-\frac{z^2}{2}} dz \\
&= \Phi(-0.5)+ \int_{-\frac{1}{2}}^{\sqrt{\frac{7}{2}}} \frac{1}{\sqrt{2\pi}}e^{-\frac{z^2}{2}} dz \\
&=0.3085+0.66082 \\
&=0.96932
\end{align}
\begin{align}
\Phi\myvec{-\sqrt{\frac{7}{2}}} &= \int_{-\infty}^{-\sqrt{\frac{7}{2}}} \frac{1}{\sqrt{2\pi}}e^{-\frac{z^2}{2}} dz \\
&=0.03068
\end{align}
\begin{align}
\pr{\myvec{X_1 - 2X_2 + 2X_3}^2 < \frac{7}{2}}&= \Phi\myvec{\sqrt{\frac{7}{2}}} - \ \Phi\myvec{-\sqrt{\frac{7}{2}}} \\
&=0.96932-0.03068 \\
&=0.93864
\end{align}
\section{Theory}
To calculate the probability $P((X_1 - 2X_2 + 2X_3)^2 < \frac{7}{2})$, where $X_1$, $X_2$, and $X_3$ are random variables with mean vector $\vec{\mu}$ = $[0, 0, 0]$ and covariance matrix $\Sigma$ given by:
\[
\Sigma = \begin{bmatrix}
2 & 2 & 1 \\
2 & 5 & 1 \\
1 & 1 & 1 \\
\end{bmatrix}
\]
and the coefficients for the linear combination $\vec{a} = [1, -2, 2]$, we can follow these steps:

\begin{enumerate}
    \item Calculate Mean and Variance of $\vec{Y}$=$\vec{X_1}-\vec{2X_2}+\vec{2X_3}$  \\
    The mean of $Y$, denoted as $\mu_Y$, is calculated as follows:
    \begin{align}
    \mu_Y &= \sum_{i=1}^{3} a_i \mu_i \\
    &= 1 \times 0 + (-2) \times 0 + 2 \times 0 \\
    &= 0
    \end{align}
    The variance of $Y$, denoted as $\sigma_Y^2$, is calculated using the formula:
    \begin{align}
    \sigma_Y^2 &= \sum_{i=1}^{3} \sum_{j=1}^{3} a_i \Sigma_{ij} a_j \\
    &= 1
    \end{align}
    
    \item Calculate Lower and Upper Bounds for Standard Normal Distribution \\
    Using the variance $\sigma_Y^2$, calculate the lower and upper bounds for the standard normal distribution corresponding to $\frac{7}{2}$:
    \begin{align}
    \text{Lower Bound} &= -\sqrt{\frac{7}{2}} \\
    \text{Upper Bound} &= \sqrt{\frac{7}{2}}
    \end{align}
    
    \item Calculate Cumulative Probabilities Using Standard Normal CDF \\
    Calculate the cumulative probabilities for the lower and upper bounds using the error function (\texttt{erf}) and the standard normal cumulative distribution function (CDF) formula:
    \begin{align}
    \text{Lower Probability} &= \frac{1}{2} \left(1 + \text{erf}\left(\frac{\text{Lower Bound}}{\sqrt{2}}\right)\right) \\
    \text{Upper Probability} &= \frac{1}{2} \left(1 + \text{erf}\left(\frac{\text{Upper Bound}}{\sqrt{2}}\right)\right)
    \end{align}
    
    \item Calculate Required Probability \\
    Calculate the required probability by subtracting the lower probability from the upper probability:
    \[
    \text{Required Probability} = \text{Upper Probability} - \text{Lower Probability}
    \]
\end{enumerate}

\section{Simulation}

The given C code calculates and prints the required probability using the calculated lower and upper probabilities.

\begin{verbatim}
#include <stdio.h>
#include <math.h>

int main() {
    // Given mean vector and covariance matrix
    double mu[3] = {0, 0, 0};
    double Sigma[3][3] = {{2, 2, 1},
                          {2, 5, 1},
                          {1, 1, 1}};

    // Coefficients for the linear combination
    double a[3] = {1, -2, 2};

    // Calculate mean and variance of $Y = X_1 - 2X_2 + 2X_3$
    double mu_Y = 0;
    double sigma_Y_squared = 0;
    for (int i = 0; i < 3; ++i) {
        mu_Y += a[i] * mu[i];
        for (int j = 0; j < 3; ++j) {
            sigma_Y_squared += a[i] * Sigma[i][j] * a[j];
        }
    }

    // Calculate the probability using standard normal CDF
    double lower_bound = -sqrt(sigma_Y_squared);
    double upper_bound = sqrt(sigma_Y_squared);

    // Calculate cumulative probabilities using math.erf for standard normal CDF
    double lower_prob = 0.5 * (1 + erf(lower_bound / sqrt(2)));
    double upper_prob = 0.5 * (1 + erf(upper_bound / sqrt(2)));

    // Calculate the required probability
    double required_prob = upper_prob - lower_prob;

    printf("P((X_1 - 2X_2 + 2X_3)^2 < 7/2) = %f\n", required_prob);

    return 0;
}
\end{verbatim}
\end{document}





